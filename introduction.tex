\section{Introduction}
	While searching for webpages online has been easy since the advent of
	the search engine, searching specifically for files has not been quite
	as intuitive.
\subsection{Current Solutions}
	\begin{multicols}{2}
		When BitTorrent \cite{bittorrent} was invented in 2001, suddenly we had a way of sharing
		files quickly, and without having to own or rent any sort of hosting
		infrastructure. The downloading of a torrent itself may be
		decentralised, yet to actually find the link to the file itself most
		users use a centralised index --- something I have always found ironic.

		The downsides of this are many:

		\begin{description}
			\item[censorship]
				Any government, organization, or other controlling and
				potentially oppressive entity has the power to block access or
				totally remove an index. This can result in a vast loss of
				information.
			\item[traffic]
				An index can see vast amounts of traffic flowing
				through it, making it expensive and difficult to run.
			\item[data loss]
				For any of the above reasons, an index may go down. It could
				even just be an administration error. If it has no backups,
				access to files could potentially be lost forever.
			\item[discovery] If you want people to be able to download your
				file, then you'd better put it on a popular index. This means
				that there are a few top indexes having a monopoly, leading to
				yet further centralisation.

	
		\end{description}
	\end{multicols}
\subsection{DFI}
\begin{multicols}{2}
	DFI offers a new approach. Instead of having a database of magnet links and
	metadata in a datacentre somewhere, why not have it everywhere? Give
	everyone access to a means of sharing, without any limits. 

	Instead of one index indexing everyone's data, take a more peer-to-peer (p2p) approach. Make
	each node in the network store a database of its own files, have them index 
	their own files. Take it a step further and allow any interested peers to
	download any other node's database, becoming both a seed for future downloads
	and a backup in case anything goes wrong. They also get the benefit of
	increased search speed. Peers can then answer both search queries for files,
	or download queries for their database. The former doesn't require a
	download step, but relies on the peer being available. The latter has greatly
	increased privacy, and doesn't rely on any other peers being online.

	Peer details --- such as name, address, public key, etc - are stored across
	the network. All that is needed is an address derived from the public key,
	and using Kademlia-style \cite{kademlia} lookups it can be resolved to the details needed.

	Adding to the address resolution, DFI can actively explore the network to
	try and increase its knowledge of other peers. 

	To aid in address resolution, peers can store a database of peer details,
	reducing their own address resolution time and overall network bandwidth
	usage. We can then allow this database to be searched by users, allowing for
	easy content discovery.
\end{multicols}

	\subsection{Searching}
	\begin{multicols}{2}
	DFI makes an important distinction between two different types of searching.
	If, for instance, we wish to search a peer's collection, it can happen in
	one of two ways.

	\begin{description}
		\item[remote] make a connection, and send them a search query.
	                  This is more like a traditional index, and has minimal 
					  friction.
		\item[local] this first requires downloading a peer's database,
		             which may take a while depending on bandwidth. However,
					 once this has been done, all searches can be performed very
					 quickly, with no remote requests needed --- which is also
					 good for privacy.

	\end{description}
\end{multicols}
	
	\subsubsection{Global search}
	\begin{multicols}{2}
	Initially, the idea of everyone having full search access to all posts all
	the time seemed like a great idea. However, systems like Gnutella
	\cite{gnutella} have
	issues with malware and spam. 

	In a decentralised system, there is obviously no centralised control. One of
	the downsides of this is a complete lack of any sort of content moderation,
	weeding out the rubbish and ensuring that the content is real.

	There is also the issue of copyrighted content spreading. 

	Overall, there would be the possibility that the network would become rife
	with fake, illegal, and generally dangerous downloads.

	With the approach DFI has taken, we have to know the peer address in order
	to search them. While we could search the peer database for keywords,
	instead of fake content being there whether we want it to be or not, users
	are in charge of moderating their own content. If a node is filled with
	spam, users can just choose not to include it with their search results in
	the first place.

	This approach has seen success in pretty much all subscription-based models.

\end{multicols}

\subsection{Encryption and anonymity}
\begin{multicols}{2}
	DFI provides no encryption, no anonymity, and does not in fact route any
	connections or packets like other systems may. It relies on an underlying
	system for this sort of thing, and permits the use of SOCKS proxies to use
	such services. Below I will detail a few networks that could be used.

	\begin{description}
		\item[Tor] \cite{tor} provides anonymity via Onion routing. It also encrypts by
			default, and aids in NAT traversal by using Onion addresses
			(.onion). However, it is slow, has high latency, and isn't entirely
			suited to p2p applications. 
		\item[I2P] \cite{i2p} is similar to Tor, though with the intention of not
			accessing the ``outer" normal internet. It also provides some level
			of anonymity and encryption. Its routing splits connections via 
			many nodes, and every node in the network contributes. While it may 
			be more suited to p2p, it has not been as studied or attacked as 
			Tor has, and is slower due to a lack of popularity compared to Tor.
			It also provides NAT traversal.
		\item[cjdns] \cite{cjdns} provides no anonymity, but it does provide an encrypted
			IPv6 network, and is very fast and reliable. It should also be
			scalable, though it has not been studied as much as Tor. NAT
			traversal is also possible.
	\end{description} 

	None of these methods of anonymity should be trusted for personal safety, especially
	against any sort of state-level actor.

	It could be best to make cjdns a standard for DFI, as it provides all of the
	needed features, as well as causing no performance hits (it may potentially
	even increase performance and reliability).

	The only downside is a lack of anonymity, though no users should \textbf{require}
	this and can gain some degree of privacy by not putting personal and
	identifying details in their \texttt{Entry}. Those needing it can use Tor.
\end{multicols}
