\section{Entries}
DFI represents peers as Entries. These contain information about a peer, and can
be used to connect to it, download a collection of the peer's posts, etc. Much
like posts, I have found SQLite to be useful for storing entries.\\

\begin{minipage}{\linewidth}
\begin{center}
\begin{tabular}{ c|p{6cm}|c }
	\textbf{Field} & \textbf{Purpose} & \textbf{Limits} \\  
	\hline
	Address        & A DFI address, as specified below \\
	\hline
	Name           & The name of this node, specified by the user & 64 chars \\
	\hline
	Desc           & A brief description of this node, like a bio & 256 chars \\
	\hline
	PublicAddress  & The address we should connect to to access this node:
	                 could be IP, domain, onion, etc
				   & 256 chars \\
	\hline
	PublicKey      & The Ed25519 public key for this \texttt{Entry}, used for verification
	               & 32 bytes \\
	\hline
	PostCount      & The number of posts this \texttt{Entry} has \\
	\hline
	Updated        & When this node last updates it's \texttt{Entry} \\
	\hline
	Signature      & The signature for this \texttt{Entry}, used to verify all details
	                 are correct. & 64 bytes \\
	\hline
	CollectionHash & The hash of a collection, defined in the next section & 32
	bytes \\
	\hline
	Port           & The port this node listens for DFI collections on & max
	65536 \\
	\hline
	Seeds          & a 2D array of bytes, each being a raw DFI address that
	seeds this node \\
	\hline
	Seeding        & a 2D array of bytes, each being a raw DFI address that this
	node is seeding \\
	\hline
	Seen           & When this node was last seen in the network \\
	\hline
	Meta           & A JSON-encoded object allowing for protocol extension \\
	\hline

\end{tabular}
\end{center}
\end{minipage}

\subsection{Address}
DFI has addresses that very much resemble Bitcoin addresses when encoded \cite{bitcoin}. In
fact, they are generated in pretty much the same way! Here is an example:
\begin{lstlisting}
	ZfkbZZXdHgLcmRoTqJjXpbZLtpWFLousS6
\end{lstlisting}

	\subsubsection{Features}
\begin{multicols}{2}
Addresses are derived from an Ed25519 public key. This means that a node can ask
any peer for the public key for a given address, and then derive an address from
it. If this derivation is successful, then we know that we have recieved the
correct public key. If not, then something is going wrong.

The binary version of an address is always 20 bytes, due to the hash function
used. However, this is not particularly human-readable, so base58check encoding
is used. This is slightly different to base64, and originally developed for use
in Bitcoin. It has the benefit of avoiding ambiguous characters such as ``0O" or
``I1". Also seeing as it's entirely alphanumeric, various UX interactions work
nicely with it.
\end{multicols}
\subsubsection{Generating an address}
\begin{multicols}{2}
Generating an address is actually very easy. DFI uses SHA3-256 and BLAKE2.
\begin{enumerate}
	\item take the SHA3-256 sum of the public key
	\item take the BLAKE2 sum of the previous sum
	\item optionally base58check encode for human readability
\end{enumerate}
So essentially, BLAKE2(SHA3-256(publicKey))

While similar to Bitcoin, DFI differs in choice of SHA hash function. SHA2 is a
Merkle-Damg\aa{}rd hash, and therefore vulnerable to a length
extension vulnerability. While a double hash solves this problem, it does
introduce some other issues. SHA3, being in the Keccak family, does not suffer
from a length extension flaw \cite{length} --- therefore making it the best choice for a new
protocol.

When base58check encoding the address, it is prefixed with a ``Z". This makes it
obvious that this is a DFI address, and not anything else. 
\end{multicols}

\newpage
\subsection{Signing}
\begin{multicols}{2}
	\texttt{Entries} must be signed in order to ensure that they can be shared and
	verified. Once the public key is known, it is easy to verify signatures.
	However, in order to do this we must first convert the \texttt{Entry} to a byte form
	in order to sign it. We can't just use JSON, as this does
	not guarantee any form of field ordering. We do not expect
	this format to ever be decoded, so no separators are needed. \texttt{Entries} 
	are encoded by first converting each field to a string, then appending each 
	string in order, like so:
\end{multicols}

	\begin{center}
	\begin{minipage}{0.5\linewidth}
	\begin{enumerate}
		\item name
		\item description
		\item public address
		\item public key
		\item port
		\item post count
		\item updated
		\item collection hash
		\item seeding \footnotemark
		\item meta
	\end{enumerate}
\end{minipage}
\end{center}
\footnotetext{Each address we are seeding is stored as raw binary, not base58
				encoded. They are also appended in the raw form.}

	\subsubsection{Mutability}
	\begin{multicols}{2}
	You may have noticed that the \texttt{Entry} does not sign all of its fields. This is
	intentional! For instance, while it only makes sense for the \texttt{Entry} to change
	what it is seeding, this is not true for who seeds the \texttt{Entry}. New seeds can
	come about at any time, and it makes sense for all seeds to maintain a list
	of all other seeds, regardless of whether or not the node itself is online.

	As well as that, signing the signature before replacing it would mean that
	entries would never validate.

	Finally, the Seen field is not signed, as this should be updated by other
	nodes. 
\end{multicols}
