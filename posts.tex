\section{Posts}
DFI is all about sharing posts. Posts represent an individual file or a
collection of files, as well as any associated metadata. They can be stored in
any way the client wishes, though I've found SQLite to be useful. It has a
full text search feature which is particularly useful, see FTS4. In memory they
can be easily represented as a struct.

Since a post contains a BitTorrent infohash, the files it is tied to can be
downloaded using any BitTorrent client. Post searches can also be ordered by
seeders and leechers, as these are indicative of popularity.\\

\begin{minipage}{\linewidth}
\begin{center}
	\begin{tabular}{ c|p{8cm} }
	\textbf{Field} & \textbf{Purpose} \\  
	\hline
	Id             & starting at one and incrementing with
	                 every post: ids are local to a collection \\
	\hline
	InfoHash       & a BitTorrent infohash, can be used to download the file \\
	\hline
	Title          & the name of the file, used for searching \\
	\hline
	Size           & the size of the file, in bytes \\
	\hline
	FileCount      & how many files this post contains \\
	\hline
	Seeders        & how many seeders the torrent file has \\
	\hline
	Leechers       & how many leechers the torrent file has \\
	\hline
	UploadDate     & the Unix timestamp (seconds) that this post was created at
	\\
	\hline
	Tags           & a comma-separated list of tags relating to this post \\
	\hline
	Meta           & a JSON object of metadata: useful for extensions\\
	\hline
\end{tabular}
\end{center}
\end{minipage}

\subsection{Pieces}
\begin{multicols}{2}
DFI stores posts in groups called Pieces. Each \texttt{Piece} contains 1000 posts, and
make it possible to verify posts. When downloading posts from a peer, it is done
a piece at a time, as pieces do not need to be full --- while they can contain
1000 posts, they could just contain one.

This carries on. Piece 0 may have 1000 posts, while piece 1 may only have 10.

A piece contains a checksum, and for this DFI uses SHA3-256. Every time a post
is added to a piece, the checksum is updated. In this way, we can
cryptographically verify that we have gotten all the posts we expect.
\end{multicols}

\subsubsection{Post hashing}
\begin{multicols}{2}
Because encoding methods such as JSON do not guarantee any sort of field
ordering, which is required when generating a checksum, Posts must be encoded by
DFI. When writing a post to a hashsum, the fields are written to a string in the
order above. However, for reasons explained in \ref{secPostImmutability}, seeders and leechers are not
included. 

For instance:
\end{multicols}
\begin{lstlisting}
Id|InfoHash|Title|Size|FileCount|UploadDate|Tags|Meta
\end{lstlisting}

\pagebreak
\subsubsection{Post immutability}
\label{secPostImmutability}
\begin{multicols}{2}
Due to the fact that we may well be downloading a post database from a peer
that did not author it, post data \textbf{must} be immutable and signed. Otherwise seeds
would be free to edit titles and descriptions wherever they please, which could
be confusing.

However, the only post field that should be able to change are the seeder and
leecher count, as generally they tend to decrease over time. Because of this,
they are not included in the hashing of a post for a piece checksum, meaning
that any peer can change them. 

When downloading a post from a seed, it would be prudent to check the
seed/leech count with a BitTorrent tracker yourself, to avoid fakes.
\end{multicols}

\subsection{Collections}
\begin{multicols*}{2}
Collections are, as the name suggests, collections of pieces. Essentially, each
node in the network authors a single collection of pieces, and this collection
may grow in time. Or, they may be empty.

Let's say Alice is the author of a collection, Eve is a seed for the
collection, and Bob is attempting to download it from Eve. Eve could modify
posts, and Bob would be none the wiser. 

This is where collections come into play. Alice takes her pieces, and checksums
each individual piece. Each checksum is then summed again, still using SHA3-256
- this forms a hashlist, with the final checksum being the root hash. Alice can
then insert the root hash into her \texttt{Entry}, and sign the \texttt{Entry} with her
private key.

Collections are made up of a root hash and a list of hashes.  Since
addresses can be used to verify the public key of an \texttt{Entry}, which can in turn be
used to verify the \texttt{Entry} itself, Bob can fetch Alice's root hash from any node.
These can then be used to fetch the hash list from anyone who is seeding Alice,
in this case Eve. The hash list can be checksummed, and the resulting checksum
compared to the root hash in Alice's \texttt{Entry}. 

If the two checksums match, then Bob knows that he has a valid hash list.

Bob can then continue to download pieces from any of Alice's seeds, safe in the
knowledge that they can all be verified via the hash list to be the same pieces
that Alice authored. 

With this in place, if Eve attempts to modify any piece or post, then checksums
will no longer match, and Bob will know.
\end{multicols*}
