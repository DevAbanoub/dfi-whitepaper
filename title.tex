\newcommand{\subtitle}[1]{%
  \posttitle{%
    \par\end{center}
    \begin{center}\large#1\end{center}
    \vskip0.5em}%
}

\fancypagestyle{titlepage}
{
	\fancyhf{}
	\renewcommand{\headrulewidth}{0pt}
	\renewcommand{\footrulewidth}{0.4pt}
}

\title{\textbf{DFI: Peer-to-Peer File Search}}
\author{Will Huxtable \\ \href{mailto:w@zif.io}{w@zif.io}}

\maketitle
\thispagestyle{titlepage}

\vspace{\fill}
\begin{abstract}
	DFI is a peer-to-peer file sharing and indexing protocol. It is primarily
	for indexing and sharing BitTorrent infohashes and any associated metadata.
	Metadata includes things such as title, upload date, and file size.  It uses
	an algorithm based on Kademlia to resolve peers, and actively tries to build
	its database of other nodes. It communicates using multiplexed TCP streams
	(similar to http/2), as these help provide NAT traversal.  Authentication
	and signing uses Curve25519 keys, and node addresses are generated and
	encoded in a similar manner to Bitcoin addresses. DFI allows peers to both
	search each other remotely, or download a copy of the database for local
	searching. The protocol does not presently offer encryption, and those
	desiring this should use something such as Tor, I2P, cjdns, etc.
\end{abstract}
\vspace{\fill}
